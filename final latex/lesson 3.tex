% ====================================================================
% A/L ICT - Lesson 3: Data Representation & Logic (Full Page Version)
% Optimized for A4 B&W Printing, Single Page
% Designed for use on Overleaf.com
% ====================================================================

% --- Document Class & Packages ---
\documentclass[a4paper, 8pt]{extarticle}

% --- Geometry for Margins ---
\usepackage[a4paper, margin=0.9cm, top=1cm, bottom=1cm]{geometry}

% --- Essential Packages ---
\usepackage{multicol}
\usepackage[svgnames]{xcolor}
\usepackage[most]{tcolorbox}
\usepackage{enumitem}
\usepackage{amssymb}
\usepackage{amsmath}
\usepackage{tabularx}

% --- Font Setup ---
\usepackage{fontspec}
\setmainfont{Inter}[
  UprightFont = *-Regular,
  BoldFont = *-Bold,
  ItalicFont = *-Italic
]
\newfontfamily{\headingfont}{Montserrat}[
  UprightFont = *-SemiBold,
  BoldFont = *-Bold
]
\setmonofont{Roboto Mono}

% --- Custom Commands for Styling ---
\newcommand{\sectionheading}[1]{%
  \par\vspace{0.7em}
  {\headingfont\fontsize{10.5pt}{11.5pt}\selectfont\color{black}#1}\par\nopagebreak
  \rule{\linewidth}{0.4pt}\vspace{0.2em}\nopagebreak
}
\newcommand{\subsectionheading}[1]{%
  \par\vspace{0.4em}\nopagebreak
  {\headingfont\fontsize{9pt}{10pt}\selectfont\color{black!80}#1}\par\nopagebreak\vspace{-0.3em}
}

% --- tcolorbox configuration ---
\tcbset{
  colback=white, colframe=black, boxrule=0.5pt,
  sharp corners, boxsep=3pt, left=4pt, right=4pt, top=3pt, bottom=3pt,
  fonttitle=\bfseries\small,
  arc=1mm
}

% --- List configuration ---
\setlist[itemize,1]{leftmargin=*, noitemsep, topsep=1.5pt, label=\textbullet}
\setlist[itemize,2]{leftmargin=*, noitemsep, topsep=1pt, label=--}
\setlist[itemize,3]{leftmargin=*, noitemsep, topsep=1pt, label=$\rightarrow$}

% --- Start of the Document ---
\begin{document}
\pagestyle{empty}

% --- Title ---
\begin{center}
    {\headingfont\fontsize{12pt}{14pt}\selectfont \textbf{Lesson 3: Data Representation \& Logic}}
\end{center}
\vspace{-0.8em}

% --- Main 2-Column Layout ---
\begin{multicols}{2}

% =============================================
% --- COLUMN 1: Concepts & Character Codes ---
% =============================================

\sectionheading{1. Number Representation Concepts}
\subsectionheading{Core Ideas}
\begin{itemize}
    \item \textbf{Why different systems?} Binary is for computers; Octal \& Hex are human-friendly shortcuts for long binary strings.
    \item \textbf{MSB/LSB:} Most/Least Significant Bit. The bit with the highest/lowest place value.
\end{itemize}

\subsectionheading{Signed Integers (Representing +/-)}
\begin{itemize}
    \item \textbf{Sign-Magnitude:} Left-most bit is the sign (0=+, 1=-). Simple, but has two zeros (+0, -0) and makes arithmetic complex.
    \item \textbf{One's Complement:} Negate by flipping all bits (0s to 1s, 1s to 0s). Still has two zeros.
    \item \textbf{Two's Complement:} Flip all bits + 1. The standard for computers because it has only one zero and makes subtraction hardware simple (subtraction becomes addition).
\end{itemize}

\subsectionheading{Decimal Number Representation}
\begin{itemize}
    \item \textbf{Fixed-point:} For numbers with a fixed number of decimal places (e.g., currency \$123.45). Simple \& fast. Limited range.
    \item \textbf{Floating-point:} For scientific numbers (very large/small). More flexible range, but more complex.
        \begin{itemize}
            \item \textbf{Structure:} Sign bit | Exponent | Mantissa.
            \item \textbf{IEEE 754 Standard:}
                \begin{itemize}
                    \item \textbf{Single Precision (32-bit):} 1 Sign, 8 Exp, 23 Man.
                    \item \textbf{Double Precision (64-bit):} 1 Sign, 11 Exp, 52 Man.
                \end{itemize}
        \end{itemize}
\end{itemize}

\sectionheading{2. Character Representation}
\subsectionheading{Character Codes \& Comparison}
\begin{itemize}
    \item \textbf{BCD (Binary Coded Decimal):}
        \begin{itemize}
            \item Represents only numbers 0-9 (4-bit).
            \item \textit{Pro: Easy decimal conversion. Con: Wastes space.}
        \end{itemize}
    \item \textbf{ASCII (American Standard...):}
        \begin{itemize}
            \item 7-bit (128 chars). Standard for English \& PCs.
            \item \textit{Pro: Widely compatible. Con: Limited characters.}
        \end{itemize}
    \item \textbf{EBCDIC (Extended BCD...):}
        \begin{itemize}
            \item 8-bit (256 chars). Used mainly by IBM Mainframes. Not compatible with ASCII's letter ordering.
        \end{itemize}
    \item \textbf{Unicode:}
        \begin{itemize}
            \item 16/32-bit. Represents all world languages.
            \item \textit{Pro: Universal standard. Con: Uses more memory than ASCII.}
        \end{itemize}
\end{itemize}

\sectionheading{3. Arithmetic \& Logic}
\subsectionheading{Binary Arithmetic}
\begin{itemize}
    \item \textbf{Addition Rules:} \texttt{0+0=0, 0+1=1, 1+1=0 carry 1}.
    \item \textbf{Subtraction Rules:} \texttt{1-1=0, 1-0=1, 0-0=0, 0-1=1 borrow 1}.
\end{itemize}
\subsectionheading{Bitwise Logic Operations}
\begin{itemize}
    \item \textbf{NOT:} Inverts bits (\texttt{NOT 1100 -> 0011}).
    \item \textbf{AND:} Masks bits (keeps bits set in \textbf{both}).
    \item \textbf{OR:} Sets bits (keeps bits set in \textbf{either}).
    \item \textbf{XOR:} Toggles bits (keeps bits that \textbf{differ}).
\end{itemize}

\columnbreak

% =============================================
% --- COLUMN 2: Rules & Examples ---
% =============================================
\sectionheading{4. Number Conversion Examples}

\begin{tcolorbox}[title=Decimal $\rightarrow$ Binary (Integer)]
\textbf{Rule:} Divide by 2, read remainders up. \\
\textbf{Ex: Convert $43_{10}$ to Binary}
\begin{verbatim}
43 / 2 = 21 R 1  ^
21 / 2 = 10 R 1  |
10 / 2 =  5 R 0  |
 5 / 2 =  2 R 1  |
 2 / 2 =  1 R 0  |
 1 / 2 =  0 R 1  |
Ans: 101011_2
\end{verbatim}
\end{tcolorbox}

\begin{tcolorbox}[title=Binary $\rightarrow$ Decimal]
\textbf{Rule:} Use place values ($...16, 8, 4, 2, 1$). \\
\textbf{Ex: Convert $101011_2$ to Decimal}
\begin{verbatim}
  1   0   1   0   1   1
  x   x   x   x   x   x
 32 + 0 + 8 + 0 + 2 + 1 = 43_10
\end{verbatim}
\end{tcolorbox}

\begin{tcolorbox}[title=Binary $\leftrightarrow$ Octal \& Hex]
\textbf{Rule:} Group bits from right (3 for Oct, 4 for Hex).
\begin{itemize}
    \item \textbf{Binary to Octal: $101110_2$}
    \begin{itemize}
        \item \texttt{101 | 110 $\rightarrow$ 5 | 6 $\rightarrow$ $56_8$}
    \end{itemize}
    \item \textbf{Binary to Hex: $10111110_2$}
    \begin{itemize}
        \item \texttt{1011 | 1110 $\rightarrow$ B | E $\rightarrow$ $\text{BE}_{16}$}
    \end{itemize}
    \item \textbf{Hex to Binary: $2\text{A}_16$}
     \begin{itemize}
        \item \texttt{2 | A $\rightarrow$ 0010 | 1010 $\rightarrow$ $00101010_2$}
    \end{itemize}
\end{itemize}
\end{tcolorbox}

\begin{tcolorbox}[title=Decimal $\rightarrow$ Binary (Fraction)]
\textbf{Rule:} Multiply fraction by 2, read integer parts down. \\
\textbf{Ex: Convert $0.8125_{10}$ to Binary}
\begin{verbatim}
0.8125 * 2 = 1.625  (1) |
0.625  * 2 = 1.25   (1) |
0.25   * 2 = 0.5    (0) v
0.5    * 2 = 1.0    (1)
Ans: 0.1101_2
\end{verbatim}
\end{tcolorbox}

\begin{tcolorbox}[title=2's Complement Example: -45 in 8-bit]
\begin{enumerate}
    \item \textbf{Positive (+45):} \texttt{00101101}
    \item \textbf{One's Complement:} \texttt{11010010} (Flip all bits)
    \item \textbf{Two's Complement:} \texttt{11010011} (Add 1 to result)
\end{enumerate}
\end{tcolorbox}

\end{multicols}
\end{document}
