% ====================================================================
% A/L ICT - Lesson 4: Logic Gates & Boolean Algebra (Enhanced Readability)
% Optimized for A4 B&W Printing, Single Page
% Designed for use on Overleaf.com
% ====================================================================

% --- Document Class & Packages ---
\documentclass[a4paper, 8pt]{extarticle}

% --- Geometry for Margins ---
\usepackage[a4paper, margin=0.9cm, top=1cm, bottom=1cm]{geometry}

% --- Essential Packages ---
\usepackage{multicol}
\usepackage[svgnames]{xcolor}
\usepackage[most]{tcolorbox}
\usepackage{enumitem}
\usepackage{amssymb}
\usepackage{amsmath}
\usepackage{tabularx}

% --- Font Setup ---
\usepackage{fontspec}
\setmainfont{Inter}[
  UprightFont = *-Regular,
  BoldFont = *-Bold,
  ItalicFont = *-Italic
]
\newfontfamily{\headingfont}{Montserrat}[
  UprightFont = *-SemiBold,
  BoldFont = *-Bold
]
\setmonofont{Roboto Mono}

% --- Custom Commands for Styling ---
\newcommand{\sectionheading}[1]{%
  \par\vspace{0.7em}
  {\headingfont\fontsize{10.5pt}{11.5pt}\selectfont\color{black}#1}\par\nopagebreak
  \rule{\linewidth}{0.4pt}\vspace{0.2em}\nopagebreak
}
\newcommand{\subsectionheading}[1]{%
  \par\vspace{0.4em}\nopagebreak
  {\headingfont\fontsize{9pt}{10pt}\selectfont\color{black!80}#1}\par\nopagebreak\vspace{-0.3em}
}

% --- tcolorbox configuration ---
\tcbset{
  colback=white, colframe=black, boxrule=0.5pt,
  sharp corners, boxsep=4pt, left=4pt, right=4pt, top=3pt, bottom=4pt, % Increased bottom boxsep
  fonttitle=\bfseries\small,
  arc=1mm
}

% --- List configuration ---
\setlist[itemize,1]{leftmargin=*, noitemsep, topsep=1.5pt, label=\textbullet}
\setlist[itemize,2]{leftmargin=*, noitemsep, topsep=1pt, label=--}
\setlist[enumerate,1]{leftmargin=*, noitemsep, topsep=2pt}

% --- Start of the Document ---
\begin{document}
\pagestyle{empty}

% --- Title ---
\begin{center}
    {\headingfont\fontsize{12pt}{14pt}\selectfont \textbf{Lesson 4: Logic Gates \& Boolean Algebra}}
\end{center}
\vspace{-0.8em}

% --- Main 3-Column Layout ---
\begin{multicols}{3}

% =============================================
% --- COLUMN 1: GATES & BOOLEAN LAWS ---
% =============================================

\sectionheading{1. Logic Gates}
\subsectionheading{Gate Lexicon}
\begin{tcolorbox}
\renewcommand{\arraystretch}{1.3} % Increase row spacing in table
\begin{tabularx}{\linewidth}{l X X}
\textbf{Gate} & \textbf{Expression} & \textbf{Output is 1 when...} \\
\hline
NOT  & $L = \overline{A}$ & input is 0. \\ \hline
AND  & $L = A \cdot B$ & all inputs are 1. \\ \hline
OR   & $L = A + B$ & at least one input is 1. \\ \hline
NAND & $L = \overline{A \cdot B}$ & any input is 0. \\ \hline
NOR  & $L = \overline{A + B}$ & all inputs are 0. \\ \hline
XOR  & $L = A \oplus B$ & inputs are different. \\ \hline
XNOR & $L = \overline{A \oplus B}$ & inputs are the same. \\
\end{tabularx}
\end{tcolorbox}

\subsectionheading{Universal Gates (NAND \& NOR)}
\begin{itemize}
    \item \textbf{Why universal?} Any logic gate can be constructed using only NAND gates or only NOR gates. This makes manufacturing ICs cheaper and simpler.
\end{itemize}
\begin{tcolorbox}[title=Implementations with NAND]
\begin{itemize}
    \item \textbf{NOT A} = $(A \cdot A)'$
    \item \textbf{A AND B} = $((A \cdot B)')'$
    \item \textbf{A OR B} = $(A' \cdot B')'$
\end{itemize}
\end{tcolorbox}

\sectionheading{2. Boolean Algebra}
\subsectionheading{Key Laws \& Theorems}
\begin{itemize}
    \item \textbf{Identity:} $A+0=A$ , $A \cdot 1=A$
    \item \textbf{Complement:} $A+\overline{A}=1$ , $A \cdot \overline{A}=0$
    \item \textbf{Commutative:} $A+B=B+A$
    \item \textbf{Associative:} $A+(B+C)=(A+B)+C$
    \item \textbf{Distributive:} $A(B+C)=AB+AC$
    \item \textbf{De Morgan's:}
    \begin{itemize}
        \item $\overline{A \cdot B} = \overline{A} + \overline{B}$
        \item $\overline{A+B} = \overline{A} \cdot \overline{B}$
    \end{itemize}
    \item \textbf{Redundancy:} $A+\overline{A}B=A+B$
\end{itemize}

\columnbreak

% =============================================
% --- COLUMN 2: SIMPLIFICATION ---
% =============================================

\sectionheading{3. Standard Forms}
\subsectionheading{SOP \& POS}
\begin{itemize}
    \item \textbf{SOP (Sum of Products):} Sum of AND terms (Minterms). Derived from rows where output is \textbf{1}.
    \begin{itemize}
        \item \textit{Ex: $F = \overline{A}BC + A\overline{B}C$}
    \end{itemize}
    \item \textbf{POS (Product of Sums):} Product of OR terms (Maxterms). Derived from rows where output is \textbf{0}.
     \begin{itemize}
        \item \textit{Ex: $F = (A+B+C)(A+\overline{B}+C)$}
    \end{itemize}
\end{itemize}

\begin{tcolorbox}[title=Deriving SOP from a Truth Table]
\renewcommand{\arraystretch}{1.2}
\begin{tabular}{ccc|c|l}
\textbf{A} & \textbf{B} & \textbf{C} & \textbf{F} & \textbf{Minterm} \\
\hline
0 & 0 & 0 & 0 & \\ \hline
0 & 0 & 1 & \textbf{1} & $\rightarrow \overline{A}\overline{B}C$ \\ \hline
0 & 1 & 0 & 0 & \\ \hline
0 & 1 & 1 & \textbf{1} & $\rightarrow \overline{A}BC$ \\ \hline
... & ... & ... & ... &
\end{tabular} \\
\vspace{0.5em}
\textbf{SOP Expression:} $F = \overline{A}\overline{B}C + \overline{A}BC$
\end{tcolorbox}

\sectionheading{4. Karnaugh Maps (K-Maps)}
\subsectionheading{Simplification Rules}
\begin{enumerate}
    \item Group only 1s, never 0s.
    \item Groups must be rectangular \& contain a power of 2 number of cells (1, 2, 4, 8...).
    \item No diagonal groups.
    \item Make groups as large as possible.
    \item Groups can overlap. Wrap-around is allowed.
    \item Use the fewest number of groups possible.
\end{enumerate}

\begin{tcolorbox}[title=3-Variable K-Map Example]
\textbf{Simplify $F = \overline{A}\overline{B}C + \overline{A}BC + A\overline{B}\overline{C} + A\overline{B}C$}
\begin{enumerate}
    \item \textbf{Map the 1s:}
    
    \begin{tabular}{r|c|c|c|c|}
      & \multicolumn{4}{c}{AB} \\
      \cline{2-5}
      C & 00 & 01 & 11 & 10 \\
      \cline{2-5}
      0 & & & & \textbf{1} \\
      \cline{2-5}
      1 & \textbf{1} & \textbf{1} & & \textbf{1} \\
      \cline{2-5}
    \end{tabular}
    
    \item \textbf{Group the 1s:}
    Two groups are made.
    \begin{itemize}
        \item Group 1 (Green): The two 1s in column AB=10. This simplifies to $A\overline{B}$.
        \item Group 2 (Blue): The two 1s in row C=1, columns AB=00 and AB=01. This simplifies to $\overline{A}C$.
    \end{itemize}
    \item \textbf{Final Expression:} $F = A\overline{B} + \overline{A}C$
\end{enumerate}
\end{tcolorbox}

\columnbreak

% =============================================
% --- COLUMN 3: CIRCUITS & APPLICATIONS ---
% =============================================

\sectionheading{5. Circuit Types}
\subsectionheading{Combinational vs. Sequential}
\begin{itemize}
    \item \textbf{Combinational Circuits:}
        \begin{itemize}
            \item Output depends \textbf{only} on the current inputs.
            \item Has \textbf{no memory}.
            \item \textit{Examples: Adders, Decoders.}
        \end{itemize}
    \item \textbf{Sequential Circuits:}
        \begin{itemize}
            \item Output depends on current inputs \textbf{and} past states.
            \item Has \textbf{memory} to store past states.
            \item Uses a \textbf{feedback loop} (output is fed back as an input).
            \item \textit{Examples: Flip-Flops, Counters.}
        \end{itemize}
\end{itemize}

\sectionheading{6. Practical Circuits}
\subsectionheading{Adders}
\begin{itemize}
    \item \textbf{Half Adder:} Adds 2 bits.
        \begin{itemize}
            \item \textbf{Sum} $= A \oplus B$
            \item \textbf{Carry} $= A \cdot B$
        \end{itemize}
    \item \textbf{Full Adder:} Adds 3 bits (A, B, $C_{in}$). Built with two Half Adders and an OR gate.
        \begin{itemize}
            \item \textbf{Sum} $= A \oplus B \oplus C_{in}$
            \item \textbf{Carry Out} $= A \cdot B + C_{in}(A \oplus B)$
        \end{itemize}
\end{itemize}

\subsectionheading{Memory}
\begin{itemize}
    \item \textbf{Flip-Flop:} The basic building block of memory. A sequential circuit that can \textbf{store} a single bit (0 or 1).
    \item \textbf{RS Latch (Flip-Flop):} Simplest type, built with cross-coupled NAND or NOR gates. Has Set (S) and Reset (R) inputs.
\end{itemize}

\begin{tcolorbox}[title=Design Example: Simple Alarm]
\textbf{Problem:} An alarm (F) should sound if a sensor (A) is triggered, but only if the system is armed (B) and the door is closed (C).
\begin{enumerate}
    \item \textbf{Truth Table:} The only time F=1 is when A=1, B=1, and C=1.
    \item \textbf{Expression:} The SOP expression is simply the minterm for that one case: $F = A \cdot B \cdot C$.
    \item \textbf{Circuit:} A single 3-input AND gate.
\end{enumerate}
\end{tcolorbox}

\end{multicols}
\end{document}
