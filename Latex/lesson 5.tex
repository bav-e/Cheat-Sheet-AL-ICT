% ====================================================================
% A/L ICT - Lesson 5: Operating Systems Cheat Sheet
% Optimized for A4 B&W Printing using LaTeX
% Designed for use on Overleaf.com
% ====================================================================

% --- Document Class & Packages ---
\documentclass[a4paper, 8pt]{extarticle}

% --- Geometry for Margins ---
\usepackage[a4paper, margin=1.2cm, top=1.5cm, bottom=1.5cm]{geometry}

% --- Essential Packages ---
\usepackage{multicol}        % For creating multi-column layouts
\usepackage[svgnames]{xcolor} % For colors (we'll use shades of gray)
\usepackage{tcolorbox}       % For creating styled boxes
\usepackage{enumitem}        % For customizing lists
\usepackage{tikz}            % For diagrams (like process states)

% --- Font Setup (for a clean, modern look) ---
\usepackage{fontspec}
\setmainfont{Inter}
\setmonofont{Roboto Mono}

% --- Custom Commands for Styling ---
\newcommand{\sectionheading}[1]{\large\textbf{#1}\par\noindent\rule{\linewidth}{0.4pt}}
\newcommand{\subsectionheading}[1]{\normalsize\textbf{#1}}
\tcbset{
  colback=gray!10, colframe=gray!75, boxrule=0.5pt,
  sharp corners, boxsep=4pt, left=4pt, right=4pt, top=4pt, bottom=4pt
}
\setlist[itemize]{leftmargin=*, noitemsep, topsep=2pt}
\setlist[enumerate]{leftmargin=*, noitemsep, topsep=2pt}

% --- Start of the Document ---
\begin{document}
\pagestyle{empty} % Removes page numbers

% --- Title ---
\begin{center}
    \fontsize{12pt}{14pt}\selectfont
    \textbf{Lesson 5: Operating Systems}
\end{center}
\vspace{1em}

% --- Main 2-Column Layout ---
\begin{multicols}{2}

% =============================================
% --- COLUMN 1: INTRO & RESOURCE MGMT ---
% =============================================

\sectionheading{1. Introduction to the OS}
\vspace{0.5em}
\begin{itemize}
    \item \textbf{Definition:} System software acting as an intermediary between user/apps and hardware.
    \item \textbf{Main Goals:} Provide a Virtual Machine (hides complexity), Manage resources, Execute applications.
\end{itemize}

\subsectionheading{Evolution Timeline}
\begin{itemize}
    \item \textbf{No OS:} Serial Processing.
    \item \textbf{Simple Batch Systems:} Used tapes, improved CPU use.
    \item \textbf{Multi-programmed:} Multiple programs in memory.
    \item \textbf{Time-Sharing:} Interactive, uses context switching.
\end{itemize}

\subsectionheading{OS Classification}
\begin{itemize}
    \item \textbf{By Tasking:} Single-user/Single-task, Single-user/Multi-task, Multi-user/Multi-task.
    \item \textbf{Multi-threading:} Improves performance by parallel execution of \textit{sub-processes}.
    \item \textbf{Real-Time (RTOS):} For systems needing precise timing and high reliability (e.g., industrial control).
\end{itemize}

\vspace{1em}
\sectionheading{4. Memory \& Device Management}
\vspace{0.5em}

\subsectionheading{Memory Management}
\begin{itemize}
    \item \textbf{Virtual Memory:} Using the disk as an extension of RAM, allowing programs to be larger than physical memory.
    \item \textbf{Paging:} Dividing logical memory into \textbf{pages} and physical memory into \textbf{frames}. Can cause \textit{Internal Fragmentation}.
    \item \textbf{MMU (Memory Management Unit):} The hardware that translates virtual addresses to physical addresses.
\end{itemize}

\subsectionheading{Device Management}
\begin{itemize}
    \item \textbf{Device Driver:} A specific software that acts as a translator between the OS and a hardware device.
    \item \textbf{Spooling (Simultaneous Peripheral Operations On-Line):} Using a buffer (on disk) to hold jobs for a slow device like a printer, freeing up the CPU.
\end{itemize}

\columnbreak % End Column 1, Start Column 2

% =============================================
% --- COLUMN 2: PROCESS & FILE MGMT ---
% =============================================

\sectionheading{2. Process \& Scheduling Mgmt}
\vspace{0.5em}
\begin{itemize}
    \item \textbf{Process vs. Program:} A \texttt{Program} is passive code; a \texttt{Process} is a "program in execution".
    \item \textbf{Process Types:} \textit{I/O Bound} (spends more time on I/O) vs. \textit{Processor Bound} (spends more time on CPU).
    \item \textbf{Process States:} New $\rightarrow$ Ready $\leftrightarrow$ Running $\rightarrow$ Terminated. Running $\leftrightarrow$ Blocked. \textit{(7-state model adds Ready/Suspend \& Blocked/Suspend states)}.
    \item \textbf{PCB (Process Control Block):} The "ID Card" for a process. Contains: State, ID, Program Counter, CPU Registers, etc.
    \item \textbf{Context Switching:} Saving one process's state (to its PCB) and loading another's to share the CPU.
    \item \textbf{Interrupt:} A signal that pauses the CPU to handle an event.
\end{itemize}

\subsectionheading{Schedulers  & Policies}
\begin{tcolorbox}[title=\textbf{Scheduler Comparison}]
\begin{itemize}
    \item \textbf{Long-term:} Admits jobs, controls degree of multiprogramming.
    \item \textbf{Short-term:} Selects next process for CPU. Fastest.
    \item \textbf{Medium-term:} Swaps processes to/from memory.
\end{itemize}
\end{tcolorbox}
\begin{itemize}
    \item \textbf{Policies:} \textit{Non-preemptive} (process runs until it stops) vs. \textit{Preemptive} (OS can force a stop).
    \item \textbf{Metrics:} Turnaround Time, Response Time, Throughput, Waiting Time.
\end{itemize}

\vspace{1em}
\sectionheading{3. File \& Storage Management}
\vspace{0.5em}
\begin{itemize}
    \item \textbf{File Views:} \textit{Logical} (user's view) vs. \textit{Physical} (OS's view on disk).
    \item \textbf{File Attributes:} Name, Type, Owner, Access Permissions, Date.
    \item \textbf{File Systems:}
        \begin{itemize}
            \item \textbf{FAT:} Older, simple.
            \item \textbf{NTFS:} Modern. Advantages: Auto error recovery, Unicode support, Security, Large disk support.
        \end{itemize}
    \item \textbf{Storage Allocation Methods:}
        \begin{itemize}
            \item \textbf{Contiguous:} One single block. \textit{Pro: Fast. Con: External Fragmentation.}
            \item \textbf{Linked:} Blocks scattered, linked by pointers. \textit{Pro: No external fragmentation. Con: Slow random access.}
            \item \textbf{Indexed:} An "index block" points to all data blocks.
        \end{itemize}
    \item \textbf{Disk Formatting:} \textit{Low-level} (basic medium prep) vs. \textit{High-level} (creating a file system).
    \item \textbf{Defragmentation:} Rearranging fragmented files back into a continuous block to improve performance.
\end{itemize}

\end{multicols}

\end{document}
