% ====================================================================
% A/L ICT - Lesson 4: Logic Gates & Boolean Algebra Cheat Sheet
% Optimized for A4 B&W Printing using LaTeX
% Designed for use on Overleaf.com
% ====================================================================

% --- Document Class & Packages ---
% 'extarticle' allows for smaller font sizes like 8pt.
\documentclass[a4paper, 8pt]{extarticle}

% --- Geometry for Margins ---
\usepackage[a4paper, margin=1.2cm]{geometry}

% --- Essential Packages ---
\usepackage{amsmath}         % For math symbols and equations
\usepackage{multicol}        % For creating multi-column layouts
\usepackage[svgnames]{xcolor} % For colors (we'll use shades of gray)
\usepackage{tcolorbox}       % For creating styled boxes
\usepackage{tabularx}        % For better tables
\usepackage{amssymb}         % For additional math symbols like \oplus

% --- Font Setup (for a clean, modern look) ---
\usepackage{fontspec}
\setmainfont{Inter}
\setmonofont{Roboto Mono}

% --- Custom Commands for Styling ---
\newcommand{\sectionheading}[1]{\large\textbf{#1}\par\noindent\rule{\linewidth}{0.4pt}}
\newcommand{\subsectionheading}[1]{\normalsize\textbf{#1}}
\tcbset{
  colback=gray!10, % Light gray background for boxes
  colframe=gray!75, % Darker gray border
  boxrule=0.5pt,
  sharp corners,
  boxsep=4pt,
  left=4pt, right=4pt, top=4pt, bottom=4pt
}

% --- Start of the Document ---
\begin{document}
\pagestyle{empty} % Removes page numbers

% --- Title ---
\begin{center}
    \fontsize{12pt}{14pt}\selectfont
    \textbf{Lesson 4: Logic Gates \& Boolean Algebra}
\end{center}
\vspace{1em}

% --- Main 3-Column Layout ---
\begin{multicols}{3}

% --- COLUMN 1: THE LOGIC GATE LEXICON ---
\sectionheading{1. Logic Gates}
\vspace{0.5em}

\begin{tabularx}{\linewidth}{|l|c|X|X|}
\hline
\textbf{Gate} & \textbf{Symbol} & \textbf{Expression} & \textbf{Trigger} \\
\hline
NOT  & \includegraphics[height=0.8em]{not.png} & $L = \overline{A}$ & Inverter \\ \hline
AND  & \includegraphics[height=1.2em]{and.png} & $L = A \cdot B$ & All inputs must be 1 \\ \hline
OR   & \includegraphics[height=1.2em]{or.png} & $L = A + B$ & At least one input is 1 \\ \hline
NAND & \includegraphics[height=1.2em]{nand.png} & $L = \overline{A \cdot B}$ & Opposite of AND \\ \hline
NOR  & \includegraphics[height=1.2em]{nor.png} & $L = \overline{A + B}$ & Opposite of OR \\ \hline
XOR  & \includegraphics[height=1.2em]{xor.png} & $L = A \oplus B$ & Inputs must be different \\ \hline
XNOR & \includegraphics[height=1.2em]{xnor.png} & $L = \overline{A \oplus B}$ & Inputs must be the same \\ \hline
\end{tabularx}

\vspace{1em}
\begin{tcolorbox}
  \subsectionheading{Universal Gates}
  \vspace{0.25em}
  \textit{NAND} and \textit{NOR} are universal because they can create any other logic gate. They are economical and easy to fabricate.
\end{tcolorbox}
\columnbreak % End Column 1, Start Column 2

% --- COLUMN 2: THE SIMPLIFICATION TOOLKIT ---
\sectionheading{2. Boolean Algebra \& Simplification}
\vspace{0.5em}

\subsectionheading{Boolean Laws}
\begin{multicols}{2}
\begin{itemize}
    \item \textbf{Idempotent:} $A+A=A$
    \item \textbf{Identity:} $A+0=A$
    \item \textbf{Inverse:} $A+\overline{A}=1$
    \item \textbf{Commutative:} $A+B=B+A$
    \item \textbf{Associative:} $A+(B+C)=(A+B)+C$
    \item \textbf{Distributive:} $A(B+C)=AB+AC$
    \item \textbf{Redundancy:} $A+\overline{A}B=A+B$
\end{itemize}
\columnbreak
\begin{itemize}
    \item \textbf{Idempotent:} $A \cdot A=A$
    \item \textbf{Identity:} $A \cdot 1=A$
    \item \textbf{Inverse:} $A \cdot \overline{A}=0$
    \item \textbf{Commutative:} $A \cdot B=B \cdot A$
    \item \textbf{Associative:} $A(BC)=(AB)C$
    \item \textbf{De Morgan's:} $\overline{A \cdot B}=\overline{A}+\overline{B}$
    \item \textbf{De Morgan's:} $\overline{A+B}=\overline{A} \cdot \overline{B}$
\end{itemize}
\end{multicols}

\subsectionheading{Standard Forms}
\begin{itemize}
    \item \textbf{SOP (Sum of Products):} Sum of AND terms. (e.g., $AB + BC$)
    \item \textbf{POS (Product of Sums):} Product of OR terms. (e.g., $(A+B)(B+C)$)
\end{itemize}

\subsectionheading{Karnaugh Map (K-Map) Rules}
\begin{enumerate}
    \item Groups must contain $2^n$ cells (1, 2, 4, 8...).
    \item Groups can be horizontal or vertical, \underline{not} diagonal.
    \item Groups should be as large as possible.
    \item All '1's must be in at least one group.
    \item Groups can overlap.
    \item Wrap-around grouping is allowed.
    \item Aim for the fewest number of groups possible.
\end{enumerate}
\columnbreak % End Column 2, Start Column 3

% --- COLUMN 3: PRACTICAL APPLICATIONS ---
\sectionheading{3. Practical Applications}
\vspace{0.5em}

\subsectionheading{Half Adder}
\begin{itemize}
    \item \textbf{Purpose:} Adds 2 bits.
    \item \textbf{Outputs:} Sum, Carry.
    \item \textbf{Expressions:} 
        \begin{itemize}
            \item[] $Sum = A \oplus B$
            \item[] $Carry = A \cdot B$
        \end{itemize}
    \item \textbf{Logic Circuit:} An XOR gate for the Sum and an AND gate for the Carry.
\end{itemize}

\vspace{1em}
\subsectionheading{Full Adder}
\begin{itemize}
    \item \textbf{Purpose:} Adds 3 bits (A, B, $C_{in}$).
    \item \textbf{Outputs:} Sum, $C_{out}$.
    \item \textbf{Expressions:}
        \begin{itemize}
            \item[] $Sum = A \oplus B \oplus C_{in}$
            \item[] $C_{out} = A \cdot B + C_{in}(A \oplus B)$
        \end{itemize}
    \item \textbf{Logic Circuit:} Can be built with two Half Adders and an OR gate.
\end{itemize}

\vspace{1em}
\begin{tcolorbox}
  \subsectionheading{Flip-Flop}
  \vspace{0.25em}
  The basic building block of memory. It is a circuit that can \textbf{store} a single bit (0 or 1).
\end{tcolorbox}

\end{multicols}
\end{document}
