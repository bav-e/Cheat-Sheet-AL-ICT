% ====================================================================
% A/L ICT - Lesson 13: New Trends and Future Directions of ICT
% Optimized for A4 B&W Printing using LaTeX
% Based on the final revised plan.
% ====================================================================

% --- Document Class & Packages ---
\documentclass[a4paper, 8pt]{extarticle}

% --- Geometry for Margins ---
\usepackage[a4paper, margin=1cm]{geometry}

% --- Essential Packages ---
\usepackage{multicol}        % For creating multi-column layouts
\usepackage[svgnames]{xcolor} % For colors (we'll use shades of gray)
\usepackage{tcolorbox}       % For creating styled boxes
\usepackage{enumitem}        % For customizing lists
\usepackage{tabularx}        % For better tables

% --- Font Setup (for a clean, modern look) ---
\usepackage{fontspec}
\setmainfont{Inter}
\setmonofont{Roboto Mono}

% --- Custom Commands for Styling ---
\newcommand{\sectionheading}[1]{\large\textbf{#1}\par\noindent\rule{\linewidth}{0.4pt}}
\newcommand{\subsectionheading}[1]{\normalsize\textbf{#1}}
\tcbset{
  colback=gray!10, colframe=gray!75, boxrule=0.5pt,
  sharp corners, boxsep=4pt, left=4pt, right=4pt, top=4pt, bottom=4pt
}
\setlist[itemize]{leftmargin=*, noitemsep, topsep=3pt, label=\textbullet}

% --- Start of the Document ---
\begin{document}
\pagestyle{empty} % Removes page numbers

% --- Title ---
\begin{center}
    \fontsize{12pt}{14pt}\selectfont
    \textbf{Lesson 13: New Trends and Future Directions of ICT}
\end{center}
\vspace{1em}

% --- Main 2-Column Layout ---
\begin{multicols}{2}

% =============================================
% --- COLUMN 1 ---
% =============================================

\sectionheading{1. "The Thinking Machine" - AI}
\vspace{0.5em}
\subsectionheading{Intelligent and Emotional Computing}
\begin{itemize}
    \item The concept of machines that can think and perceive emotions.
\end{itemize}

\subsectionheading{Artificial Intelligence (AI)}
\begin{itemize}
    \item \textbf{Definition:} The simulation of human intelligence processes by machines.
    \item \textbf{Strong AI (AGI):} An AI system that can think and have a mind, capable of working in multiple fields. (Theoretical).
    \item \textbf{Weak AI (Narrow AI):} An AI system that only pretends to think, excelling in a single, narrow task. \textit{Ex: IBM Deep Blue (Chess)}.
\end{itemize}

\begin{tcolorbox}[title=\textbf{Key AI Techniques}]
    \begin{itemize}
        \item \textbf{Search Techniques:} Finding a goal state in a state space.
        \item \textbf{Expert Systems:} Rule-based (If-Then) systems that store knowledge to advise humans.
        \item \textbf{NLP:} Algorithms to recognize/understand human languages.
        \item \textbf{Machine Learning:} Techniques to learn hidden patterns from data.
        \item \textbf{Neural Networks:} A key ML technique based on artificial neurons.
        \item \textbf{Genetic Algorithms:} Optimization based on evolution.
        \item \textbf{Fuzzy Logic:} Control systems based on "fuzzy" linguistic statements (e.g., 'hot') instead of binary true/false.
    \end{itemize}
\end{tcolorbox}

\subsectionheading{Coexistence}
\begin{itemize}
    \item \textbf{Man-Machine:} Humans and intelligent machines working together.
    \item \textbf{Machine-to-Machine:} Intelligent machines communicating and acting without human intervention.
\end{itemize}

\vspace{1em}
\sectionheading{3. "Future Computing" - Beyond von Neumann}
\vspace{0.5em}
\subsectionheading{Beyond von-Neumann Computer}
\begin{itemize}
    \item \textbf{Reason:} Traditional architecture is reaching physical limits (e.g., heat), as described by Moore's Law.
\end{itemize}

\subsectionheading{Nature-Inspired Computing}
\begin{itemize}
    \item Algorithms modeled on natural phenomena.
    \item \textbf{Examples:} Genetic Algorithms, Neural Networks, \textbf{Swarm Intelligence} (ant colonies), \textbf{Membrane Computing} (living cells).
\end{itemize}

\subsectionheading{Bio-Inspired Computing}
\begin{itemize}
    \item Computing models based on biological systems (e.g., DNA, the brain). Closely related to Nature-Inspired Computing.
\end{itemize}

\subsectionheading{Quantum Computing}
\begin{itemize}
    \item \textbf{Fundamentals:} Based on quantum mechanics, using Qubits which can be in a state of \textbf{Superposition}.
    \item \textbf{Applications:} Solving complex problems, drug discovery, materials science.
\end{itemize}


\columnbreak % End Column 1, Start Column 2

% =============================================
% --- COLUMN 2 ---
% =============================================

\sectionheading{2. "The Autonomous Agent" - Agent Technology}
\vspace{0.5em}
\subsectionheading{Software Agents}
\begin{itemize}
    \item \textbf{Definition:} Software that acts on behalf of a user/program, working autonomously and continuously.
    \item \textbf{Characteristics:} Autonomous, Proactive, Reactive, Cooperative, Able to learn, Social ability.
\end{itemize}

\subsectionheading{Multi-Agent Systems (MAS)}
\begin{itemize}
    \item \textbf{Definition:} A system of multiple interacting agents solving a complex problem.
    \item \textbf{Characteristics:} Agents are autonomous, have a \textbf{local view} (only know their part), and are \textbf{decentralized}.
    \item \textbf{Common Architecture:} 
    \begin{itemize}
        \item \textbf{Interface Agent:} Connects user to the system.
        \item \textbf{Broker Agent:} Filters information.
        \item \textbf{Information Agents:} Gather data from sources.
    \end{itemize}
\end{itemize}

\subsectionheading{Applications of Agent Systems}
\begin{itemize}
    \item Virtual assistants (Siri, Cortana), e-commerce search, online booking systems.
\end{itemize}


\end{multicols}
\end{document}
