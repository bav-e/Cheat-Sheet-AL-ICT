% ====================================================================
% A/L ICT - Lesson 11: Embedded Systems & IoT Cheat Sheet
% Optimized for A4 B&W Printing using LaTeX
% Based on the final revised plan.
% ====================================================================

% --- Document Class & Packages ---
\documentclass[a4paper, 8pt]{extarticle}

% --- Geometry for Margins ---
\usepackage[a4paper, margin=1cm]{geometry}

% --- Essential Packages ---
\usepackage{multicol}        % For creating multi-column layouts
\usepackage[svgnames]{xcolor} % For colors (we'll use shades of gray)
\usepackage{tcolorbox}       % For creating styled boxes
\usepackage{enumitem}        % For customizing lists
\usepackage{tabularx}        % For better tables

% --- Font Setup (for a clean, modern look) ---
\usepackage{fontspec}
\setmainfont{Inter}
\setmonofont{Roboto Mono}

% --- Custom Commands for Styling ---
\newcommand{\sectionheading}[1]{\large\textbf{#1}\par\noindent\rule{\linewidth}{0.4pt}}
\newcommand{\subsectionheading}[1]{\normalsize\textbf{#1}}
\tcbset{
  colback=gray!10, colframe=gray!75, boxrule=0.5pt,
  sharp corners, boxsep=4pt, left=4pt, right=4pt, top=4pt, bottom=4pt
}
\setlist[itemize]{leftmargin=*, noitemsep, topsep=3pt, label=\textbullet}

% --- Start of the Document ---
\begin{document}
\pagestyle{empty} % Removes page numbers

% --- Title ---
\begin{center}
    \fontsize{12pt}{14pt}\selectfont
    \textbf{Lesson 11: Embedded Systems \& IoT}
\end{center}
\vspace{1em}

% --- Main 2-Column Layout ---
\begin{multicols}{2}

% =============================================
% --- COLUMN 1 ---
% =============================================

\sectionheading{1. The "Thing" - Embedded Systems \& Arduino}
\vspace{0.5em}
\subsectionheading{Core Concepts}
\begin{itemize}
    \item \textbf{Embedded System:} A computer system inside a larger system. Also known as \textbf{Physical Computing}.
    \item \textbf{IPO Model:} Input (Sensors) $\rightarrow$ Process (Microcontroller) $\rightarrow$ Output (Actuators).
    \item \textbf{Microcontroller vs. Microprocessor:} A microcontroller is a single chip with integrated CPU, memory, and I/O. A microprocessor-based system has these as separate components.
    \item \textbf{IEEE Definition:} A computer system that is part of a larger system and performs some of its requirements.
\end{itemize}

\subsectionheading{The Development Board: Arduino Uno}
\begin{itemize}
    \item \textbf{Key Features:} USB Port (for programming/power), Analog Pins (A0-A5), Microcontroller (ATmega328P), Digital I/O Pins (0-13).
\end{itemize}

\subsectionheading{The Software: Arduino IDE}
\begin{itemize}
    \item Arduino programs are called \textbf{Sketches}.
    \item \textbf{IDE Components:} Verify, Upload, Code Editor, Console Window.
\end{itemize}

\subsectionheading{Firmware Structure \& Logic}
\begin{itemize}
    \item Must have two main functions: \texttt{void setup()} (runs once) and \texttt{void loop()} (runs over and over).
    \item An \textbf{endless loop} is used because there is no OS to return control to.
\end{itemize}

\subsectionheading{Commonly Used Components (Practicals)}
\begin{itemize}
    \item \textbf{Outputs:} LED, Piezo Buzzer, DC Motor.
    \item \textbf{Inputs:} LDR, LM35 Temp Sensor, Reed Switch.
    \item \textbf{Other:} Resistor (\texttt{220$\Omega$}, \texttt{10k$\Omega$}), Transistor (\texttt{BC547}), Diode (\texttt{1N4001}).
\end{itemize}

\begin{tcolorbox}[title=\textbf{Core Arduino Functions}]
    \begin{itemize}
        \item \texttt{pinMode(pin, MODE);} - Sets pin as \texttt{INPUT} or \texttt{OUTPUT}.
        \item \texttt{digitalWrite(pin, STATE);} - Sets pin to \texttt{HIGH} or \texttt{LOW}.
        \item \texttt{digitalRead(pin);} - Reads state (\texttt{HIGH}/\texttt{LOW}) from pin.
        \item \texttt{analogRead(pin);} - Reads value (0-1023) from analog pin.
        \item \texttt{delay(ms);} - Pauses program for milliseconds.
        \item \texttt{tone(pin, frequency);} - Generates a sound tone.
        \item \texttt{noTone(pin);} - Stops the tone.
        \item \texttt{Serial.begin(baudRate);} - Starts serial communication.
    \end{itemize}
\end{tcolorbox}


\columnbreak % End Column 1, Start Column 2

% =============================================
% --- COLUMN 2 ---
% =============================================

\sectionheading{2. The "Network" - Practical IoT with Arduino}
\vspace{0.5em}
\subsectionheading{Core Concepts}
\begin{itemize}
    \item \textbf{IoT Definition:} A network of interconnected embedded systems communicating over the Internet.
    \item \textbf{Goal:} To create a "Smart World" for convenience and comfort.
\end{itemize}

\subsectionheading{Building an IoT Device (Smart Light Example)}
\begin{itemize}
    \item \textbf{Key Hardware:} \textbf{Arduino Ethernet Shield} - an add-on module for internet connectivity.
    \item \textbf{Important:} When the Ethernet Shield is used, digital pins 4, 10, 11, 12, and 13 are reserved and cannot be used for other purposes.
\end{itemize}

\begin{tcolorbox}[title=\textbf{Core IoT/Ethernet Functions}]
    \begin{itemize}
        \item \texttt{\#include <Ethernet2.h>} - Include library at the start.
        \item \texttt{byte mac[] = \{...\};} - Holds the shield's MAC address.
        \item \texttt{EthernetServer s(80);} - Creates a server on port 80.
        \item \textbf{In \texttt{setup()}:}
        \begin{itemize}
            \item \texttt{Ethernet.begin(mac);} - Connects to network (DHCP).
            \item \texttt{Serial.println(Ethernet.localIP());} - Displays IP address.
            \item \texttt{server.begin();} - Starts the web server.
        \end{itemize}
        \item \textbf{In \texttt{loop()}:}
        \begin{itemize}
            \item \texttt{EthernetClient c = s.available();} - Checks for a client.
            \item \texttt{c.connected()} - Checks if client is still connected.
            \item \texttt{c.read()} - Reads data sent from the client.
            \item \texttt{c.stop()} - Disconnects the client.
        \end{itemize}
    \end{itemize}
\end{tcolorbox}

\vspace{1em}
\sectionheading{3. Applications \& Implications}
\vspace{0.5em}
\subsectionheading{Example Systems (from Practical Book)}
\begin{itemize}
    \item \textbf{Blinker:} Blinks an LED at a regular interval.
    \item \textbf{AutoLight:} Controls an LED based on LDR light intensity.
    \item \textbf{AutoFan:} Controls a motor based on LM35 temperature.
    \item \textbf{Door-Alarm:} Triggers a buzzer using a reed switch.
    \item \textbf{Smart Light (IoT):} Controls an LED remotely via HTTP.
\end{itemize}

\subsectionheading{Social \& Security Consequences}
\begin{itemize}
    \item Social Isolation.
    \item \textbf{Security \& Privacy:} Risk of unauthorized control of devices and access to personal data.
\end{itemize}


\end{multicols}
\end{document}
