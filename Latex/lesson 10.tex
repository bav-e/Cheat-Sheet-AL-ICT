% ====================================================================
% A/L ICT - Lesson 10: Web Development Cheat Sheet
% Optimized for A4 B&W Printing using LaTeX
% Based on the final revised plan.
% ====================================================================

% --- Document Class & Packages ---
\documentclass[a4paper, 8pt]{extarticle}

% --- Geometry for Margins ---
\usepackage[a4paper, margin=1cm]{geometry}

% --- Essential Packages ---
\usepackage{multicol}        % For creating multi-column layouts
\usepackage[svgnames]{xcolor} % For colors (we'll use shades of gray)
\usepackage{tcolorbox}       % For creating styled boxes
\usepackage{enumitem}        % For customizing lists
\usepackage{tabularx}        % For better tables

% --- Font Setup (for a clean, modern look) ---
\usepackage{fontspec}
\setmainfont{Inter}
\setmonofont{Roboto Mono}

% --- Custom Commands for Styling ---
\newcommand{\sectionheading}[1]{\large\textbf{#1}\par\noindent\rule{\linewidth}{0.4pt}}
\newcommand{\subsectionheading}[1]{\normalsize\textbf{#1}}
\tcbset{
  colback=gray!10, colframe=gray!75, boxrule=0.5pt,
  sharp corners, boxsep=4pt, left=4pt, right=4pt, top=4pt, bottom=4pt
}
\setlist[itemize]{leftmargin=*, noitemsep, topsep=3pt, label=\textbullet}

% --- Start of the Document ---
\begin{document}
\pagestyle{empty} % Removes page numbers

% --- Title ---
\begin{center}
    \fontsize{12pt}{14pt}\selectfont
    \textbf{Lesson 10: Web Development (HTML, CSS, PHP)}
\end{center}
\vspace{1em}

% --- Main 2-Column Layout ---
\begin{multicols}{2}

% =============================================
% --- COLUMN 1: FRONTEND DEVELOPMENT ---
% =============================================

\sectionheading{1. Frontend Development (HTML \& CSS)}
\vspace{0.5em}
\subsectionheading{HTML Core Structure \& Tags}
\begin{itemize}
    \item \textbf{Basic Structure:} \texttt{<!DOCTYPE html>}, \texttt{<html>}, \texttt{<head>}, \texttt{<title>}, \texttt{<body>}.
    \item \textbf{Comments:} \texttt{<!-- comment -->}.
    \item \textbf{Text Formatting:} \texttt{<h1>-<h6>}, \texttt{<p>}, \texttt{<br>}, \texttt{<b>}, \texttt{<i>}, \texttt{<u>}.
    \item \textbf{Links:} \texttt{<a href="url">Link Text</a>}.
    \item \textbf{Images:} \texttt{<img src="path" alt="text">}.
    \item \textbf{Lists:} \texttt{<ul>} (Unordered), \texttt{<ol>} (Ordered), \texttt{<li>} (List Item).
    \item \textbf{Tables:} 
        \begin{itemize}
            \item \textit{Core:} \texttt{<table>}, \texttt{<tr>}, \texttt{<th>}, \texttt{<td>}.
            \item \textit{Structure:} \texttt{<thead>}, \texttt{<tbody>}, \texttt{<tfoot>}.
            \item \textit{Merging:} \texttt{colspan}, \texttt{rowspan}.
        \end{itemize}
    \item \textbf{Multimedia:} \texttt{<audio>}, \texttt{<video>} with \texttt{src} attribute.
\end{itemize}

\subsectionheading{CSS Core Concepts}
\begin{itemize}
    \item \textbf{Syntax:} \texttt{selector \{ property: value; \}}.
    \item \textbf{Selectors:} Element (\texttt{p}), Class (\texttt{.name}), ID (\texttt{\#name}).
    \item \textbf{Insertion Methods:}
        \begin{itemize}
            \item \textbf{Inline:} \texttt{style} attribute in an HTML tag.
            \item \textbf{Internal:} \texttt{<style>} tag in the \texttt{<head>}.
            \item \textbf{External:} \texttt{<link>} tag or \texttt{@import} directive.
        \end{itemize}
    \item \textbf{Common Properties:} \texttt{color}, \texttt{font-size}, \texttt{font-family}, \texttt{background-color}, \texttt{margin}, \texttt{padding}, \texttt{border}.
\end{itemize}

\vspace{1em}
\sectionheading{3. Publishing \& Key Concepts}
\vspace{0.5em}
\begin{itemize}
    \item \textbf{Website Types:} Static vs. Dynamic.
    \item \textbf{Core Concepts:} Client-Server Model, URL Components (protocol, domain, path).
    \item \textbf{Publishing:} Local vs. Internet.
    \item \textbf{Key Terms:} ISP, Web Hosting, Domain Name.
    \item \textbf{Maintenance:} Importance of regular backups, security updates, and monitoring site statistics.
    \item \textbf{Authoring Tools:} Software like Dreamweaver can auto-generate code.
\end{itemize}


\columnbreak % End Column 1, Start Column 2

% =============================================
% --- COLUMN 2: BACKEND DEVELOPMENT ---
% =============================================

\sectionheading{2. Backend Development (PHP \& MySQL)}
\vspace{0.5em}
\subsectionheading{PHP Basics}
\begin{itemize}
    \item \textbf{Embedding:} \texttt{<?php ... ?>}.
    \item \textbf{Variables \& Output:} \texttt{\$variable = "value";}, \texttt{echo \$variable;}.
    \item \textbf{Comments:} \texttt{//} (single-line), \texttt{/* ... */} (multi-line).
    \item \textbf{Control Structures:} \texttt{if...elseif...else}, \texttt{switch}, \texttt{while}, \texttt{do...while}, \texttt{for}.
    \item \textbf{Array Sorting:} Functions like \texttt{sort()}, \texttt{rsort()}, \texttt{asort()} (by value), \texttt{ksort()} (by key).
\end{itemize}

\subsectionheading{HTML Forms (User to Server Bridge)}
\begin{itemize}
    \item \textbf{Main Tag:} \texttt{<form action="..." method="post|get">}.
    \item \textbf{GET vs. POST:} GET sends data in URL; POST hides it in the request body.
    \item \textbf{Input Elements:} \texttt{<input type="text">}, \texttt{"password"}, \texttt{"radio"}, \texttt{"checkbox"}, \texttt{"submit"}.
    \item \textbf{Other Elements:} \texttt{<textarea>}, \texttt{<select>}, \texttt{<label>}, \texttt{<fieldset>}.
\end{itemize}

\subsectionheading{PHP \& MySQL (Database Connection)}
\begin{tcolorbox}[title=\textbf{Typical DB Operation Flow (Procedural)}]
    \begin{enumerate}
        \item \texttt{\$link = mysqli\_connect(...);} \textit{// Establish connection}
        \item \texttt{\$sql = "INSERT INTO ...";} \textit{// Formulate SQL query}
        \item \texttt{mysqli\_query(\$link, \$sql);} \textit{// Execute query}
        \item \texttt{\$row = \$result->fetch\_assoc();} \textit{// Process results (for SELECT)}
        \item \texttt{mysqli\_close(\$link);} \textit{// Close connection}
    \end{enumerate}
\end{tcolorbox}
\begin{itemize}
    \item \textbf{Retrieving Form Data:} Use superglobals \texttt{\$\_POST['input\_name']} or \texttt{\$\_GET['input\_name']}.
\end{itemize}

\end{multicols}
\end{document}
