% ====================================================================
% A/L ICT - Lesson 8: Database Design & Development Cheat Sheet
% Optimized for A4 B&W Printing using LaTeX
% ERD-Focused. Designed for use on Overleaf.com
% ====================================================================

% --- Document Class & Packages ---
\documentclass[a4paper, 8pt]{extarticle}

% --- Geometry for Margins ---
\usepackage[a4paper, margin=1cm]{geometry}

% --- Essential Packages ---
\usepackage{multicol}        % For creating multi-column layouts
\usepackage[svgnames]{xcolor} % For colors (we'll use shades of gray)
\usepackage{tcolorbox}       % For creating styled boxes
\usepackage{enumitem}        % For customizing lists
\usepackage{graphicx}        % To scale images/symbols if needed

% --- Font Setup (for a clean, modern look) ---
\usepackage{fontspec}
\setmainfont{Inter}
\setmonofont{Roboto Mono}

% --- Custom Commands for Styling ---
\newcommand{\sectionheading}[1]{\large\textbf{#1}\par\noindent\rule{\linewidth}{0.4pt}}
\newcommand{\subsectionheading}[1]{\normalsize\textbf{#1}}
\tcbset{
  colback=gray!10, colframe=gray!75, boxrule=0.5pt,
  sharp corners, boxsep=4pt, left=4pt, right=4pt, top=4pt, bottom=4pt
}
\setlist[itemize]{leftmargin=*, noitemsep, topsep=3pt, label=\textbullet}

% --- Start of the Document ---
\begin{document}
\pagestyle{empty} % Removes page numbers

% --- Title ---
\begin{center}
    \fontsize{12pt}{14pt}\selectfont
    \textbf{Lesson 8: Database Design \& Development}
\end{center}
\vspace{1em}

% --- Main 2-Column Layout ---
\begin{multicols}{2}

% =============================================
% --- COLUMN 1 ---
% =============================================

\sectionheading{1. Database Fundamentals}
\vspace{0.5em}
\begin{itemize}
    \item \textbf{Core Concepts:} Data vs. Information; Structured vs. Unstructured Data.
    \item \textbf{Database:} An organized collection of structured information.
    \item \textbf{Database Models (Evolution):}
    \begin{itemize}
        \item \textbf{Flat-file:} Simple text files. \textit{Problem: High redundancy.}
        \item \textbf{Hierarchical:} Tree-like structure (one-to-many).
        \item \textbf{Network:} Graph-like structure (many-to-many).
        \item \textbf{Relational:} Modern standard; data in tables.
        \item \textbf{Object-Relational:} Combines Relational + OO concepts.
    \end{itemize}
\end{itemize}

\vspace{1em}
\sectionheading{2. Conceptual Design (ER & EER)}
\vspace{0.5em}
\subsectionheading{ER Diagrams (The Blueprint)}
\begin{itemize}
    \item \textbf{Entity:} A real-world object. (\textit{Symbol: Rectangle})
    \item \textbf{Attribute:} A property of an entity. (\textit{Symbol: Oval})
    \item \textbf{Key Attribute:} An underlined attribute.
    \item \textbf{Relationship:} How entities are connected. (\textit{Symbol: Diamond})
\end{itemize}
\subsectionheading{Relationship Cardinality}
\begin{itemize}
    \item \textbf{One-to-One (1:1)}
    \item \textbf{One-to-Many (1:M)}
    \item \textbf{Many-to-Many (M:M)}
\end{itemize}
\subsectionheading{EER (Extended ER) Diagrams}
\begin{itemize}
    \item Used for more complex designs involving \textbf{specialization} and \textbf{generalization}.
\end{itemize}

\vspace{1em}
\sectionheading{4. Essential SQL Commands}
\vspace{0.5em}
\begin{tcolorbox}[title=\textbf{SQL Sub-languages}]
    \begin{itemize}
        \item \textbf{DDL (Data Definition Language):} Defines the structure.
        \item \textbf{DML (Data Manipulation Language):} Manages the data.
    \end{itemize}
\end{tcolorbox}
\subsectionheading{Essential DDL Commands}
\begin{itemize}
    \item \texttt{CREATE TABLE}
    \item \texttt{ALTER TABLE}
    \item \texttt{DROP TABLE}
\end{itemize}
\subsectionheading{Essential DML Commands}
\begin{itemize}
    \item \texttt{INSERT INTO ... VALUES ...}
    \item \texttt{UPDATE ... SET ... WHERE ...}
    \item \texttt{DELETE FROM ... WHERE ...}
\end{itemize}
\subsectionheading{The Core \texttt{SELECT} Query}
\begin{itemize}
    \item \texttt{SELECT} cols \texttt{FROM} table \texttt{WHERE} condition
    \item \textit{Key Clauses:} \texttt{ORDER BY}, \texttt{GROUP BY}, \texttt{INNER JOIN}
\end{itemize}

\columnbreak % End Column 1, Start Column 2

% =============================================
% --- COLUMN 2 ---
% =============================================

\sectionheading{3. Logical Design & Normalization}
\vspace{0.5em}
\subsectionheading{Relational Schema (The Plan)}
\begin{itemize}
    \item \textbf{Relation:} Table
    \item \textbf{Tuple:} Row / Record
    \item \textbf{Attribute:} Column / Field
\end{itemize}
\subsectionheading{Key Types (The Rules)}
\begin{itemize}
    \item \textbf{Candidate Key:} Can uniquely identify a row.
    \item \textbf{Primary Key (PK):} The chosen candidate key. \textit{Cannot be NULL, must be UNIQUE.}
    \item \textbf{Alternate Key:} A candidate key that was \underline{not} chosen as the PK.
    \item \textbf{Foreign Key (FK):} A PK from one table used in another to create a link.
\end{itemize}

\subsectionheading{Normalization (Quality Control)}
\begin{itemize}
    \item \textbf{Purpose:} To reduce \textbf{data redundancy} and avoid \textbf{anomalies} (errors during Insert, Update, Delete).
\end{itemize}
\begin{tcolorbox}[title=\textbf{Functional Dependencies}]
\begin{itemize}
    \item \textbf{Partial Dependency:} A non-key attribute depends on only a \textit{part} of a composite PK.
    \item \textbf{Transitive Dependency:} A non-key attribute depends on \textit{another non-key attribute}.
\end{itemize}
\end{tcolorbox}
\subsectionheading{The Normal Forms}
\begin{itemize}
    \item \textbf{1NF (First Normal Form):}
    \begin{itemize}
        \item No repeating groups.
        \item All values must be atomic (indivisible).
    \end{itemize}
    \item \textbf{2NF (Second Normal Form):}
    \begin{itemize}
        \item Must be in 1NF.
        \item \underline{No} partial dependencies.
    \end{itemize}
    \item \textbf{3NF (Third Normal Form):}
    \begin{itemize}
        \item Must be in 2NF.
        \item \underline{No} transitive dependencies.
    \end{itemize}
\end{itemize}

\end{multicols}
\end{document}
