# ** Plan for Lesson 7 Cheat Sheet: Systems Analysis & Design**

### **The Goal**

To create a comprehensive, single-page reference sheet that covers all core concepts of systems, information system types, and the entire System Development Life Cycle (SDLC), including the specific details and terminologies from the syllabus, making it ideal for answering MCQ questions.

### **The Structure: "Following the SDLC Lifecycle"**

The cheat sheet will be structured into four logical sections that follow the natural progression of developing a system, from basic concepts to final deployment and maintenance.

## **Final Content List for the Cheat Sheet**

### **Part 1: The Foundation \- System Concepts & Development Models**

* **System Concepts:**  
  * **Definition:** Interrelated components working together to achieve an objective.  
  * **Classifications:** Open vs. Closed, Natural vs. Man-made, Living vs. Physical.  
* **Types of Information Systems (with Management Level):**  
  * **TPS (Transaction Processing System):** For *Operational Level* (daily routine transactions).  
  * **MIS (Management Information System):** For *Management Level* (routine summary reports).  
  * **DSS (Decision Support System):** For *Management Level* (semi-structured decisions).  
  * **ESS (Executive Support System):** For *Strategic Level* (unstructured decisions).  
  * **OAS (Office Automation System):** Increases office worker productivity.  
  * **KMS (Knowledge Management System):** Manages organizational knowledge.  
  * **ERP (Enterprise Resource Planning):** Integrates all business functions.  
  * **Expert System:** AI-based system that mimics a human expert.  
* **System Development Life Cycle (SDLC) Models (with Applicability):**  
  * **Waterfall:** Sequential model. *Best for projects with clear and stable requirements.*  
  * **Spiral:** Combines iteration with risk analysis. *Best for large, complex, high-risk projects.*  
  * **Agile:** Iterative, with rapid delivery of small features. *Best for projects with changing requirements.*  
  * **Prototyping:** Building a working model to get user feedback early.  
  * **RAD (Rapid Application Development):** Develops modules in parallel for fast delivery.  
* **Development Methodologies:**  
  * **Structured:** Traditional, top-down approach (e.g., SSADM).  
  * **Object-Oriented:** Models the system as a collection of interacting objects.

### **Part 2: The Beginning \- Planning & Analysis**

* **Preliminary Investigation:**  
  * **Feasibility Study (Is this project possible?):**  
    * **T**echnical, **E**conomic (Cost-Benefit), **O**perational, **O**rganizational.  
* **Requirement Analysis (What should the system do?):**  
  * **Functional Requirements:** The *activities* the system must perform.  
  * **Non-functional Requirements:** The *qualities* or *constraints* of the system.  
  * **IEEE Standard:** Essential requirements use *"Shall"*; desirable ones use *"Should"*.  
* **Analysis & Modeling Tools (SSADM Concepts):**  
  * **DFD (Data Flow Diagram):** Shows how data moves and is processed.  
    * **Components:** External Entity, Process, Data Flow, Data Store.  
    * **Levels:** Context Diagram (Level 0\) & Level 1 DFD.  
    * Includes Document Flow Diagram.  
  * **LDM (Logical Data Modeling):** Shows how data is structured.  
    * **Components:** Entity, Attribute, Relationship (One-to-One, One-to-Many, etc.).  
  * **Other Key Terms:** Elementary Process Descriptions (EPD), Business System Options (BSO).

### **Part 3: The Middle \- Design & Testing**

* **System Design:**  
  * **Logical Design:** What the system must do (tech-independent).  
  * **Physical Design:** How the system will be implemented (with tech).  
  * **Database Mapping:** Entity ➔ Table, Attribute ➔ Field.  
  * **Data Dictionary:** A central repository of metadata.  
* **System Testing (Does it work correctly?):**  
  * **Testing Techniques:**  
    * **White-box Testing:** Tests internal code structure.  
    * **Black-box Testing:** Tests functionality without seeing the code.  
  * **Testing Levels & Performers:**  
    *   
      1. **Unit Testing:** Testing individual components (*usually by programmers*).  
    *   
      2. **Integration Testing:** Testing if units work together.  
    *   
      3. **System Testing:** Testing the complete system (*usually by a separate test team*).  
    *   
      4. **Acceptance Testing:** Users test if it meets their needs.

### **Part 4: The End Game \- Deployment & Alternatives**

* **Deployment / Changeover Methods (How to go live?):**  
  * **Parallel:** Old and New run together. *Safest, most expensive.*  
  * **Direct (Big Bang):** Stop Old, start New. *Riskiest, cheapest.*  
  * **Phased:** Implement New system in parts.  
  * **Pilot:** Implement full New system for a small group first.  
* **Maintenance:** Correcting errors, improving performance after deployment.  
* **Alternative: COTS vs. Custom Software:**  
  * **COTS (Commercial-Off-The-Shelf):** Buying a ready-made package.  
    * *Pros:* Cheaper, faster. *Cons:* May not fit perfectly, no competitive advantage.  
  * **Custom Developed:** Building from scratch.  
    * *Pros:* Perfect fit, provides competitive advantage, becomes a business asset. *Cons:* Expensive, time-consuming.  
  * **Key Concepts:** Gap Analysis, Business Process Reengineering.